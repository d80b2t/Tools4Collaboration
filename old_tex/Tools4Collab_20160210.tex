\documentclass{article}
\usepackage{amsmath, amssymb}
\usepackage{fancyhdr}
\usepackage{times, hyperref}
% \usepackage[letterpaper,left=.5in,right=.5in,top=.75in,bottom=.5in]{geometry} % Uncomment for narrow margins

\date{today}

\begin{document}

\fancyhead[R]{Nic Ross et al.}
\pagestyle{fancy}

\noindent
This is just a wee ``living'' document that I wanted to get going, that basically writes down all the collaboration tools that ``the kool kids'' are using\footnote{The very fact I used that phrase, and spelt cool with a `k', kinda shows' I'm not ;-)}.  
A lot of the descriptions will be from 
\href{https://en.wikipedia.org/wiki/Main_Page}{Wikipedia}.

\noindent
{\bf N.B. This is highly incomplete and very much a work in progress at this time...}



\section{Code Repositories}

\subsection{Bitbucket}
\href{https://bitbucket.org/}{https://bitbucket.org/}\\
``Bitbucket is the Git solution for professional teams.'' 
(words from the website). 

\subsection{GitHub}
\href{https://github.com/}{https://github.com/} \\
GitHub is a Web-based Git repository hosting service. It offers all of the distributed revision control and source code management (SCM) functionality of Git as well as adding its own features. Unlike Git, which is strictly a command-line tool, GitHub provides a Web-based graphical interface and desktop as well as mobile integration. It also provides access control and several collaboration features such as bug tracking, feature requests, task management, and wikis for every project.[3]

\section{LaTeX}
Comparisons of (La)TeX editors can be found 
\href{https://en.wikipedia.org/wiki/Comparison_of_TeX_editors}{here}. 

\subsection{ShareLatex}
\href{https://www.sharelatex.com/}{https://www.sharelatex.com/} is being used right now. 

\subsection{Authorea}
\href{https://www.authorea.com/}{https://www.authorea.com/}\\
Authorea is the collaborative editor for research. Write and manage your documents in one place, for free. (Words from their website).

%From Andy: also check out 
\subsection{Overleaf}
\href{https://www.overleaf.com/}{Overleaf} (used to be WriteLaTeX) 
Overleaf is an online LaTeX and Rich Text collaborative writing and publishing tool that makes the whole process of writing, editing and publishing scientific documents much quicker and easier (from the website). 

\subsection{Papeeria}
\href{https://www.papeeria.com/landing}{Papeeria} 

\subsection{TeXPad Connect}
\href{https://www.texpadapp.com/texpad-connect}{TeXPad Connect}



\section{Notetaking}
\subsection{Evernote}
Evernote is a cross-platform, freemium app designed for note taking, organizing, and archiving. It is developed by the Evernote Corporation, a private company headquartered in Redwood City, California. The app allows users to create a "note" which can be a piece of formatted text, a full webpage or webpage excerpt, a photograph, a voice memo, or a handwritten "ink" note. Notes can also have file attachments. Notebooks can be added to a stack while notes can be sorted into a notebook, tagged, annotated, edited, given comments, searched, and exported as part of a notebook.

\section{Version Control}
\subsection{SVN}
Apache Subversion (often abbreviated SVN, after the command name svn) is a software versioning and revision control system distributed as free software under the Apache License.[2] Software developers use Subversion to maintain current and historical versions of files such as source code, web pages, and documentation. Its goal is to be a mostly compatible successor to the widely used Concurrent Versions System (CVS).

\section{File hosting services}
\subsection{DropBox}
Dropbox is a file hosting service operated by Dropbox, Inc., headquartered in San Francisco, California, that offers cloud storage, file synchronization, personal cloud, and client software.

\section{Bibliography}
Comparison of reference management software \href{https://en.wikipedia.org/wiki/Comparison_of_reference_management_software}{wikipage}. 

\subsection{Mendeley}
\href{https://www.mendeley.com/}{https://www.mendeley.com/}\\
Mendeley is a free reference manager and academic social network. Make your own fully-searchable library in seconds, cite as you write, and read and annotate your PDFs on any device. Showcase your work on your profile and assess the impact of your research. 

\subsection{Papers}
\href{http://www.papersapp.com/}{http://www.papersapp.com/}\\
Papers is a reference management software for Mac OS X and Windows,[3] used to manage bibliographies and references when writing essays and articles. It is primarily used to organize references and maintain a library of PDF documents and also provides a uniform interface for document repository searches, metadata editing, full screen reading and a variety of ways to import and export documents.



\section{Wikis}
\subsection{twiki}



\end{document}
